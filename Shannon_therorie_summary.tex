%%%%%%%%%%%%%%%%%%%%%%%%%%%%%%%%%%%%%%%%%%%%%%%%%%%
%% LaTeX book template                           %%
%% Author:  Amber Jain (http://amberj.devio.us/) %%
%% License: ISC license                          %%
%%%%%%%%%%%%%%%%%%%%%%%%%%%%%%%%%%%%%%%%%%%%%%%%%%%

\documentclass[a4paper,11pt]{report}
\usepackage[T1]{fontenc}
\usepackage[utf8]{inputenc}
\usepackage{lmodern}
%%%%%%%%%%%%%%%%%%%%%%%%%%%%%%%%%%%%%%%%%%%%%%%%%%%%%%%%%
% Source: http://en.wikibooks.org/wiki/LaTeX/Hyperlinks %
%%%%%%%%%%%%%%%%%%%%%%%%%%%%%%%%%%%%%%%%%%%%%%%%%%%%%%%%%
\usepackage{hyperref}
\usepackage{graphicx}
\usepackage[english]{babel}
\linespread{1.6}

%%%%%%%%%%%%%%%%%%%%%%%%%%%%%%%%%%%%%%%%%%%%%%%%%%%%%%%%%%%%%%%%%%%%%%%%%%%%%%%%
% 'dedication' environment: To add a dedication paragraph at the start of book %
% Source: http://www.tug.org/pipermail/texhax/2010-June/015184.html            %
%%%%%%%%%%%%%%%%%%%%%%%%%%%%%%%%%%%%%%%%%%%%%%%%%%%%%%%%%%%%%%%%%%%%%%%%%%%%%%%%
\newenvironment{dedication}
{
   \cleardoublepage
   \thispagestyle{empty}
   \vspace*{\stretch{1}}
   \hfill\begin{minipage}[t]{0.66\textwidth}
   \raggedright
}
{
   \end{minipage}
   \vspace*{\stretch{3}}
   \clearpage
}

%%%%%%%%%%%%%%%%%%%%%%%%%%%%%%%%%%%%%%%%%%%%%%%%
% Chapter quote at the start of chapter        %
% Source: http://tex.stackexchange.com/a/53380 %
%%%%%%%%%%%%%%%%%%%%%%%%%%%%%%%%%%%%%%%%%%%%%%%%
\makeatletter
\renewcommand{\@chapapp}{}% Not necessary...
\newenvironment{chapquote}[2][2em]
  {\setlength{\@tempdima}{#1}%
   \def\chapquote@author{#2}%
   \parshape 1 \@tempdima \dimexpr\textwidth-2\@tempdima\relax%
   \itshape}
  {\par\normalfont\hfill--\ \chapquote@author\hspace*{\@tempdima}\par\bigskip}
\makeatother

%%%%%%%%%%%%%%%%%%%%%%%%%%%%%%%%%%%%%%%%%%%%%%%%%%%
% First page of book which contains 'stuff' like: %
%  - Book title, subtitle                         %
%  - Book author name                             %
%%%%%%%%%%%%%%%%%%%%%%%%%%%%%%%%%%%%%%%%%%%%%%%%%%%

% Book's title and subtitle
\title{\Huge \textbf{Invitation à la théorie de l'information, Emmanuel Dion}  
%\footnote{Notes de lecture} 
\\ 
\huge Notes de lecture
%\footnote{Notes : Arabella Brayer}
}
% Author
\author{\textsc{Arabella Brayer}
%\thanks{\url{www.example.com}}
}


\begin{document}

%\frontmatter
\maketitle

%%%%%%%%%%%%%%%%%%%%%%%%%%%%%%%%%%%%%%%%%%%%%%%%%%%%%%%%%%%%%%%
% Add a dedication paragraph to dedicate your book to someone %
%%%%%%%%%%%%%%%%%%%%%%%%%%%%%%%%%%%%%%%%%%%%%%%%%%%%%%%%%%%%%%%
%\begin{dedication}
%Dedicated to Calvin and Hobbes.
%\end{dedication}

%%%%%%%%%%%%%%%%%%%%%%%%%%%%%%%%%%%%%%%%%%%%%%%%%%%%%%%%%%%%%%%%%%%%%%%%
% Auto-generated table of contents, list of figures and list of tables %
%%%%%%%%%%%%%%%%%%%%%%%%%%%%%%%%%%%%%%%%%%%%%%%%%%%%%%%%%%%%%%%%%%%%%%%%
\tableofcontents
%\listoffigures
%\listoftables

%%%%%%%%%%%
% Preface %
%%%%%%%%%%%
\chapter{Introduction}

\section*{Le concept d'information}
Le terme "information" désigne une notion difficile à décrire de façon simple 
et sans emphase, 
ou sans user d'évidences qui n'apportent aucune information utile.
Pour ce faire, on peut s'inspirer de l'analogie entre 
l'information et l'énergie, notion aux multiples formes également.
D'autre part, remarquons que de tout temps, la plupart des inventions 
ont servi à maîtriser l'une ou l'autre : énergie, information.
Quelques exemples : la radio, le téléphone, l'informatique, etc.

\section*{Épistémologie}
Du point de vue de l'épistémologie, on peut également rapprocher 
l'information de l'énergie. On constatera alors que les deux ont été 
employées avant de savoir les définir de façon formelle. 
C'est avec la théorie de Shannon que l'information a acquis un sens précis, 
ainsi qu'une unité de mesure : le bit.
C'est la parution du livre de Shannon en 1948 qui marque ce tournant, 
et qui restera dans l'histoire des sciences du XX$^{o}$. Dès ce moment, 
un nombre important de publications sortent à ce sujet, 
et la recherche clarifie son discours. 

Actuellement, la densité de travaux s'est certes un peu tarie, 
néanmoins l'ensemble de ces travaux sont rassemblés derrière l'expression 
"théorie de l'information" (ainsi que "théorie de la communication"
\footnote{"Théorie de la communication" est une expression qui désigne 
la même chose strictement, contrairement à ce que laisse entendre son 
nom. Shannon lui-même aurait préféré l'usage de l'expression 
"théorie de l'information".}
) et est largement reconnue.

Parmi les théories existantes en sciences, on pourrait trouver des éléments 
similaires entre la théorie de l'information et la théorie des jeux : 
double composante mathématique et conceptuelle, ainsi qu'une large 
diffusion. D'ailleurs, même si le lien entre ces deux théories ne 
saute pas à la conscience, elles entretiennent des relations, 
qui seront détaillées plus tard.

\section*{Utilisations de la théorie}
% You might want to add short description about each chapter in this book.
La théorie de l'information a été vue de façon différente dans la science : 
ainsi a-t-elle apporté à plusieurs domaines, tels que la biologie, la psychologie, etc. 
Mais son caractère "généraliste" lui a "permis" d'être largement citée en philosophie. 
Il s'agirait plutôt d'un emploi abusif. On pourrait tenter de réduire 
la théorie de l'information à quelques opérateurs mathématiques, 
déjà connus, mais réunis dans cette théorie. On peut également la voir 
comme une théorie primordiale pour le XX$^o$ siècle.

\textbf{Problématique} :
Ce débat a-t-il lieu d'être ou pourrait-on imaginer que ces 
deux propositions ne se rassemblent ? 

%%%%%%%%%%%%%%%%
% NEW CHAPTER! %
%%%%%%%%%%%%%%%%
\chapter{La théorie de l'information : une théorie transversale au cœur de la science moderne}

%\begin{chapquote}{Author's name, \textit{Source of this quote}}
%``This is a quote and I don't know who said this.''
%\end{chapquote}

\section{Section heading}

La théorie de l'information ne s'intéresse absolument pas à la 
signification, au sens, contrairement aux autres théories en 
communication, focalisées sur cet aspect.
Weaver et Shannon n'ont jamais souhaité donner une aura autre 
que technique à cette théorie, rappelons que cette époque est 
celle où l'on souhaite améliorer la qualité des transmissions.
Les débordements sémantiques n'ont sans doute pas lieu d’être
et surtout, ne sont pas du fait de ces deux personnes.

\subsection{Les racines de la théorie}
L'origine de la théorie vient du besoin de délimiter les 
capacités de transmission d'un message, soit par 
l'intermédiaire du canal de communication directement, 
soit par son système de codage. Différents systèmes binaires 
avaient déjà vu le jour à divers endroits du globe.
Ces systèmes possèdent des caractéristiques intéressantes, 
comme la possibilité d'employer les combinatoires, et 
d'avoir des propriétés au codage. Le morse est "efficace" à 
85\%, bien qu'inventé vers 1830, ce qui est très bien.
Construire un code efficace nécessite une théorie sur les 
fréquences d'apparition des lettres (~1300), des digrammes (~1600), 
et celles-ci n'étaient pas encore réunies. 

\subsection{L'approche statistique : l'information de Fisher}
Fisher a commencé à considérer l'information comme une quantité
mesurable, vers 1920. Il la définit comme étant 
la valeur moyenne du carré de la dérivée du logarithme de la 
loi de probabilité étudiée.

\subsection{L'approche des ingénieurs : les travaux de Nyquist et Hartley}

Parallèlement, en 1922, on trouve des premières pistes pour améliorer la qualité 
et vitesse de transmission des signaux radio.
La formule \\$W = K.logM$ résume ici que l'on considère le caractère comme unité, 
K étant une constante dépendant de la qualité de la ligne. On note le log, 
dont on reparlera. Il faut attendre 1948 pour que Shannon fasse progresser la matière.

\subsection{L'apport de Shannon}
L'objectif de Shannon est avant tout d'améliorer les 
rendements des lignes de télégraphe. 
Shannon n'est pas un grand érudit mathématique, il résout magnifiquement 
des problèmes complexes mais pratiques, plus qu'abstraits. 
C'est un homme humble, honnête intellectuel, scientifique. Son article 
déclenche de grands mouvements scientifiques, mais il reste tel qu'il est, 
préoccupé par des problèmes d'une priorité discutable. 

\subsection{Le MIT, plaque tournante du développement des sciences de l'information}

Shannon est d'abord élève, puis professeur au MIT, ce qui va beaucoup l'influencer. 
Il rencontre Wiener, et les deux se citent régulièrement l'un l'autre dans 
leurs travaux. 
Wiener et Shannon arrivent à des conclusions similaires en partant de 
deux problématiques légèrement différentes. Wiener arrive à quantifier 
la quantité d'information par 
$log_{2} \frac{quantité\_a\_priori}{quantité\_a\_posteriori}$
Wiener étend sa définition, et la rapproche de Von Neumann, de distribution continue 
de probabilité : $ \int f(x).log_{2}f(x).dx$

Il ne faut pas oublier qu'à l'époque, on écrit déjà des programmes sur cartes perforées, 
et l'on dispose d'appareils déjà évolués capable de résoudre des problèmes 
complexes comme extraction d'une racine carrée, etc. Finalement, 
la technique était en avance sur la théorie.
 
%%%%%%%%%%%%%%%%%%%%%%%%%%%%%%%%%%%%%%%%%%%%%%%%%%%%%%%
% Sample table                                        %
% Source: www1.maths.leeds.ac.uk/latex/TableHelp1.pdf %
%%%%%%%%%%%%%%%%%%%%%%%%%%%%%%%%%%%%%%%%%%%%%%%%%%%%%%%
%\begin{table}[ht]
%\caption{Sample table} % title of Table
%\centering % used for centering table
%\begin{tabular}{c c c c}
% centered columns (4 columns)
%\hline\hline %inserts double horizontal lines
%S. No. & Column\#1 & Column\#2 & Column\#3 \\ [0.5ex]
% inserts table
%heading
%\hline % inserts single horizontal line
%1 & 50 & 837 & 970 \\
%2 & 47 & 877 & 230 \\
%3 & 31 & 25 & 415 \\
%4 & 35 & 144 & 2356 \\
%5 & 45 & 300 & 556 \\ [1ex] % [1ex] adds vertical space
%\hline %inserts single line
%\end{tabular}
%\label{table:nonlin} % is used to refer this table in the text
%\end{table}

\end{document}
