%%%%%%%%%%%%%%%%%%%%%%%%%%%%%%%%%%%%%%%%%%%%%%%%%%%
%% LaTeX book template                           %%
%% Author:  Amber Jain (http://amberj.devio.us/) %%
%% License: ISC license                          %%
%%%%%%%%%%%%%%%%%%%%%%%%%%%%%%%%%%%%%%%%%%%%%%%%%%%

\documentclass[a4paper,11pt]{report}
\usepackage[T1]{fontenc}
\usepackage[utf8]{inputenc}
\usepackage{lmodern}
%%%%%%%%%%%%%%%%%%%%%%%%%%%%%%%%%%%%%%%%%%%%%%%%%%%%%%%%%
% Source: http://en.wikibooks.org/wiki/LaTeX/Hyperlinks %
%%%%%%%%%%%%%%%%%%%%%%%%%%%%%%%%%%%%%%%%%%%%%%%%%%%%%%%%%
\usepackage{hyperref}
\usepackage{graphicx}
\usepackage[english]{babel}
\linespread{1.6}

%%%%%%%%%%%%%%%%%%%%%%%%%%%%%%%%%%%%%%%%%%%%%%%%%%%%%%%%%%%%%%%%%%%%%%%%%%%%%%%%
% 'dedication' environment: To add a dedication paragraph at the start of book %
% Source: http://www.tug.org/pipermail/texhax/2010-June/015184.html            %
%%%%%%%%%%%%%%%%%%%%%%%%%%%%%%%%%%%%%%%%%%%%%%%%%%%%%%%%%%%%%%%%%%%%%%%%%%%%%%%%
\newenvironment{dedication}
{
   \cleardoublepage
   \thispagestyle{empty}
   \vspace*{\stretch{1}}
   \hfill\begin{minipage}[t]{0.66\textwidth}
   \raggedright
}
{
   \end{minipage}
   \vspace*{\stretch{3}}
   \clearpage
}

%%%%%%%%%%%%%%%%%%%%%%%%%%%%%%%%%%%%%%%%%%%%%%%%
% Chapter quote at the start of chapter        %
% Source: http://tex.stackexchange.com/a/53380 %
%%%%%%%%%%%%%%%%%%%%%%%%%%%%%%%%%%%%%%%%%%%%%%%%
\makeatletter
\renewcommand{\@chapapp}{}% Not necessary...
\newenvironment{chapquote}[2][2em]
  {\setlength{\@tempdima}{#1}%
   \def\chapquote@author{#2}%
   \parshape 1 \@tempdima \dimexpr\textwidth-2\@tempdima\relax%
   \itshape}
  {\par\normalfont\hfill--\ \chapquote@author\hspace*{\@tempdima}\par\bigskip}
\makeatother

%%%%%%%%%%%%%%%%%%%%%%%%%%%%%%%%%%%%%%%%%%%%%%%%%%%
% First page of book which contains 'stuff' like: %
%  - Book title, subtitle                         %
%  - Book author name                             %
%%%%%%%%%%%%%%%%%%%%%%%%%%%%%%%%%%%%%%%%%%%%%%%%%%%

% Book's title and subtitle
\title{\Huge \textbf{Invitation à la théorie de l'information, Emmanuel Dion}  
%\footnote{Notes de lecture} 
\\ 
\huge Notes de lecture
%\footnote{Notes : Arabella Brayer}
}
% Author
\author{\textsc{Arabella Brayer}
%\thanks{\url{www.example.com}}
}


\begin{document}

%\frontmatter
\maketitle

%%%%%%%%%%%%%%%%%%%%%%%%%%%%%%%%%%%%%%%%%%%%%%%%%%%%%%%%%%%%%%%
% Add a dedication paragraph to dedicate your book to someone %
%%%%%%%%%%%%%%%%%%%%%%%%%%%%%%%%%%%%%%%%%%%%%%%%%%%%%%%%%%%%%%%
%\begin{dedication}
%Dedicated to Calvin and Hobbes.
%\end{dedication}

%%%%%%%%%%%%%%%%%%%%%%%%%%%%%%%%%%%%%%%%%%%%%%%%%%%%%%%%%%%%%%%%%%%%%%%%
% Auto-generated table of contents, list of figures and list of tables %
%%%%%%%%%%%%%%%%%%%%%%%%%%%%%%%%%%%%%%%%%%%%%%%%%%%%%%%%%%%%%%%%%%%%%%%%
\tableofcontents
%\listoffigures
%\listoftables

%%%%%%%%%%%
% Preface %
%%%%%%%%%%%
\chapter{Introduction}

\section*{Le concept d'information}
Le terme "information" désigne une notion difficile à décrire de façon simple 
et sans emphase, 
ou sans user d'évidences qui n'apportent aucune information utile.
Pour ce faire, on peut s'inspirer de l'analogie entre 
l'information et l'énergie, notion aux multiples formes également.
D'autre part, remarquons que de tout temps, la plupart des inventions 
ont servi à maîtriser l'une ou l'autre : énergie, information.
Quelques exemples : la radio, le téléphone, l'informatique, etc.

\section*{Épistémologie}
Du point de vue de l'épistémologie, on peut également rapprocher 
l'information de l'énergie. On constatera alors que les deux ont été 
employées avant de savoir les définir de façon formelle. 
C'est avec la théorie de Shannon que l'information a acquis un sens précis, 
ainsi qu'une unité de mesure : le bit.
C'est la parution du livre de Shannon en 1948 qui marque ce tournant, 
et qui restera dans l'histoire des sciences du XX$^{o}$. Dès ce moment, 
un nombre important de publications sortent à ce sujet, 
et la recherche clarifie son discours. 

Actuellement, la densité de travaux s'est certes un peu tarie, 
néanmoins l'ensemble de ces travaux sont rassemblés derrière l'expression 
"théorie de l'information" (ainsi que "théorie de la communication"
\footnote{"Théorie de la communication" est une expression qui désigne 
la même chose strictement, contrairement à ce que laisse entendre son 
nom. Shannon lui-même aurait préféré l'usage de l'expression 
"théorie de l'information".}
) et est largement reconnue.

Parmi les théories existantes en sciences, on pourrait trouver des éléments 
similaires entre la théorie de l'information et la théorie des jeux : 
double composante mathématique et conceptuelle, ainsi qu'une large 
diffusion. D'ailleurs, même si le lien entre ces deux théories ne 
saute pas à la conscience, elles entretiennent des relations, 
qui seront détaillées plus tard.

\section*{Utilisations de la théorie}
% You might want to add short description about each chapter in this book.
La théorie de l'information a été vue de façon différente dans la science : 
ainsi a-t-elle apporté à plusieurs domaines, tels que la biologie, la psychologie, etc. 
Mais son caractère "généraliste" lui a "permis" d'être largement citée en philosophie. 
Il s'agirait plutôt d'un emploi abusif. On pourrait tenter de réduire 
la théorie de l'information à quelques opérateurs mathématiques, 
déjà connus, mais réunis dans cette théorie. On peut également la voir 
comme une théorie primordiale pour le XX$^o$ siècle.

\textbf{Problématique} :
Ce débat a-t-il lieu d'être ou pourrait-on imaginer que ces 
deux propositions ne se rassemblent ? 

%%%%%%%%%%%%%%%%
% NEW CHAPTER! %
%%%%%%%%%%%%%%%%
\chapter{La théorie de l'information : une théorie transversale au cœur de la science moderne}

%\begin{chapquote}{Author's name, \textit{Source of this quote}}
%``This is a quote and I don't know who said this.''
%\end{chapquote}

\section{Section heading}

La théorie de l'information ne s'intéresse absolument pas à la 
signification, au sens, contrairement aux autres théories en 
communication, focalisées sur cet aspect.
Weaver et Shannon n'ont jamais souhaité donner une aura autre 
que technique à cette théorie, rappelons que cette époque est 
celle où l'on souhaite améliorer la qualité des transmissions.
Les débordements sémantiques n'ont sans doute pas lieu d’être
et surtout, ne sont pas du fait de ces deux personnes.

\section{Les racines de la théorie}
L'origine de la théorie vient du besoin de délimiter les 
capacités de transmission d'un message, soit par 
l'intermédiaire du canal de communication directement, 
soit par son système de codage. Différents systèmes binaires 
avaient déjà vu le jour à divers endroits du globe.
Ces systèmes possèdent des caractéristiques intéressantes, 
comme la possibilité d'employer les combinatoires, et 
d'avoir des propriétés au codage. Le morse est "efficace" à 
85\%, bien qu'inventé vers 1830, ce qui est très bien.
Construire un code efficace nécessite une théorie sur les 
fréquences d'apparition des lettres (~1300), des digrammes (~1600), 
et celles-ci n'étaient pas encore réunies. 

\section{L'approche statistique : l'information de Fisher}
Fisher a commencé à considérer l'information comme une quantité
mesurable, vers 1920. Il la définit comme étant 
la valeur moyenne du carré de la dérivée du logarithme de la 
loi de probabilité étudiée.

\section{L'approche des ingénieurs : les travaux de Nyquist et Hartley}

Parallèlement, en 1922, on trouve des premières pistes pour améliorer la qualité 
et vitesse de transmission des signaux radio.
La formule \\$W = K\times\log{M}$ résume ici que l'on considère le caractère comme unité, 
K étant une constante dépendant de la qualité de la ligne. On note le log, 
dont on reparlera. Il faut attendre 1948 pour que Shannon fasse progresser la matière.

\section{L'apport de Shannon}
L'objectif de Shannon est avant tout d'améliorer les 
rendements des lignes de télégraphe. 
Shannon n'est pas un grand érudit mathématique, il résout magnifiquement 
des problèmes complexes mais pratiques, plus qu'abstraits. 
C'est un homme humble, honnête intellectuel, scientifique. Son article 
déclenche de grands mouvements scientifiques, mais il reste tel qu'il est, 
préoccupé par des problèmes d'une priorité discutable. 

\section{Le MIT, plaque tournante du développement des sciences de l'information}

Shannon est d'abord élève, puis professeur au MIT, ce qui va beaucoup l'influencer. 
Il rencontre Wiener, et les deux se citent régulièrement l'un l'autre dans 
leurs travaux. 
Wiener et Shannon arrivent à des conclusions similaires en partant de 
deux problématiques légèrement différentes. Wiener arrive à quantifier 
la quantité d'information par 
$log_{2}\frac{quantité-a-priori}{quantité-a-posteriori}$
Wiener étend sa définition, et la rapproche de Von Neumann, de distribution continue 
de probabilité : $ \int f(x)\times \log_{2}{f(x)}\times dx$

Il ne faut pas oublier qu'à l'époque, on écrit déjà des programmes sur cartes perforées, 
et l'on dispose d'appareils déjà évolués capable de résoudre des problèmes 
complexes comme extraction d'une racine carrée, etc. Finalement, 
la technique était en avance sur la théorie.

À cette époque, la logique de Boole n'est pas associée à l'informatique, 
ni même Turing. La référence était plutôt Von Neumann (lié à Goldstein), 
directeur du secteur mathématique d'IBM.

Les logiciens de l'époque ne s'intéressent pas à l'informatique. 

On peut citer David Slepian, comme contributeur important également : 
créateur des codes correcteurs) Peter Elias, David Huffman, Warren McCullough 
(Research Lboratory of Electronics).

\section{Un débat scientifique animé et ouvert}

Le livre de Shannon donne lieu à des controverses. 
D'abord, sur l'emploi des mots : Information, entropie, bruit, cybernétique... 
L'usage du terme "information" concernant la théorie de l'information 
peut laisser supposer que l'on s'intéresse à la communication du sens, 
mais il n'en est rien. On peut dire autrement : 
"le logarithme du maximum de vraisemblance d'une distribution multinomiale", 
on s'aperçoit qu'il n'y a pas lieu d'en faire de la philosophie. Cela a pourtant 
été fait, mais on ne peut pas le reprocher à Shannon, qui n'a eu de 
cesse de rappeler que ce n'était pas là l'ambition de sa théorie. Notons que 
les critiques ont toujours été faites quant à la théorie elle-même et non 
à l'endroit de Shannon.

Concernant l'entropie, le choix du mot renvoie forcément à un terme précis en 
thermo-dynamique, et l'on peut se demander si ce choix de mot est judicieux. 
Précisons que Shannon a écrit sa théorie sans mesurer l'impact qu'elle pourrait avoir 
par la suite, et a peut-être sous-évalué l'importance du choix des mots.

D'autres mots s'ajoutent, comme le bruit, la redondance, etc. ouvrant la porte 
à toutes sortes d'idées plus large. Mais insistons bien : la théorie de l'information 
de Shannon porte sur quelque chose d'assez technique finalement, qui est 
la communication dans un canal "physique". 
On peut se demander si sa théorie aurait eu le succès qu'elle a eu si les mots 
choisis n'avaient pas permis toutes ces extrapolations extravagantes.


Citons deux "camps" : W. Weaver, L. Brillouin, E. Jaynes, M. Tribus, E. Schoffeniels, 
T. Stonier, très "favorables".

L. Cronbach, H. Quastler, B. Mandelbrot, A. Lwoff, D. McKay, C. Waddington, R. Thom, 
etc. vont - quant à eux - contester les interprétations "abusives".


\section{Une opposition qui porte sur des thèmes fondamentaux}

Au regard des critiques, on pourrait penser que cette querelle est stérile, 
mais il n'en est rien. En réalité, ces critiques apportent beaucoup du point 
de vue de l'épistémologie. On peut bien voir un rapport entre l'information 
et l'ADN/ARN, etc.

La théorie de l'information rassemble également deux courants qui s'affrontent 
chez les probabilistes : le mouvement des fréquentistes, et des subjectivistes. 
Les fréquentistes croient qu'une expérience se doit d'être répétée afin de 
pouvoir modéliser les statistiques, l'autre pense que l'on peut théoriser le 
hasard pour analyser un modèle. On comprend que cela suscite alors des 
discussions animées.

\section{Les aspects épistémologiques du problème}

Entre la naissance de la théorie et les recherches, il s'est écoulé environ 
5 à 6 ans. Depuis, quelques personnes continuent de s'intéresser au sujet, 
mais l'âge d'or est passé.
Références : F. Resa, J. Wolfowitz, P. Elias, A. Kolmogorov (utilise l'entropie 
comme concept de base pour la classification des systèmes dynamiques).

La définition même de l'entropie n'a que peu d'importance dans le cadre rhétorique. 
Que l'on puisse en parler - point de vue rhétorique - n'apporte rien à la compréhension 
mathématique du sujet. La chose se complique lorsqu'il faut quantifier.

On peut rapprocher la théorie de l'information de ce point de vue avec la théorie des jeux, car les deux trouvent difficilement des opportunités physiques réelles de réalisation :
la théorie des jeux suppose de posséder des matrices de choix... Ce que concrètement, 
l'on n'a jamais. La théorie de l'information peut éventuellement porter sur 
quelque chose de physique, comme la génétique, 
mais au delà, on dispose peu de conditions compatibles avec la théorie. 
La preuve ne peut donc passer pour justification physique. 
D'où le rapprochement avec Godel, Von Neumann, etc., 
dans la percée de ces théories dont la déduction est à l'origine d'une réponse 
efficace/inefficace au monde physique.

\chapter{Des opérateurs mathématiques d'une grande élégance}

\section{La mesure de l'information : pourquoi le logarithme ?}

La théorie ne comporte rien de très complexe sur le plan 
de la formalisation mathématique. Il faut maîtriser deux définitions :
la quantité d'information, et l'entropie.
On peut - après ça, s'intéresser à la redondance ou au bruit mais 
ces premières sont plus importantes pour la compréhension 
du sens de la théorie.

L'information, chez Shannon, désigne un (ensemble d') événement(s) parmi 
un ensemble d'événements possibles.
Toutes les mesures qui caractérisent ces événements sont probabilistes.

Prenons comme exemple la recherche d'un livre dans une bibliothèque.
Si l'on connaît une information ``pertinente'', cela peut réduire le temps
de recherche. Par exemple, si l'on sait qu'on recherche un livre avec une 
couverture bleue, et qu'il y en a 1/4 dans la bibliothèque, c'est une 
information importante.

On cherche donc à quantifier l'information comme ce qui réduit l'incertitude.
Dans l'exemple cité avant, on a l'intuition que les nombres des livres concernés 
peuvent modifier l'information : plus de livres au total, plus de livres bleus, etc.
Mais Shannon utilise pourtant le logarithme : 


$qté\_d\_infos = I = log(\frac{N}{n})$, ce qui permet de conserver les propriétés 
additives du logarithme. 
(rappel : $\log{(a \times b)} = \log{a} + \log{b}$ )
Concernant la base du logarithme, si la base 2 a été choisie, c'est pour une raison 
principalement arbitraire : log(2) = 1 en base 2, or, il a été fixé que l'information 
valait 1 lorsqu'il y avait dichotomie parfaite. Le logarithme est un choix judicieux : 
positivité, additivité, et la base 2 pour indiquer la dichotomie parfaite.


On voit cependant que toutes les informations ne sont peut-être pas équiprobables :
en langue française, la fréquence des lettres est inégales. Dans ce cas, l'information 
sera $ I = \log{\frac{1}{p}} $ où p est le degré d'apparition, soit $ I = -\log{p} $.

Cette unité sera appelée le \textit{bit} par Shannon (porte d'autres noms pour d'autres).
Un bit peut se définir de cette façon :

\textit{la quantité d'information qui correspond à la réduction de moitié de  l'incertitude sur un problème donné}


Reprenons l'exemple de la bibliothèque et étudions l'information : 
Mettons qu'il y ait 4000 livres, et 500 bleus. L'information ``le livre recherché est bleu'' devient :
$\log{(\frac{4000}{500})} = \log{8} = 3 $ : 3 bits. On peut expliquer ça comme ça : 
on a divisé 4000 par 8 ($\frac{4000}{500} = 8$) or pour écrire de 0 à 7, il faut 3 bits en binaire.
Pour savoir dans quel tas chercher, on a donc besoin de 3 bits.

Si on avait eu 1000 livres bleus, on aurait eu besoin de 2 bits, car 
$\log{4} = 2$.
L'information est de moindre ``valeur'' dans ce cas, car les ``tas'' seront de 1000 livres...

On le voit dans cet exemple : la théorie de l'information est purement quantitative. 
On peut aussi se demander la quantité d'information I(bleu clair) contenue 
dans l'affirmation "le livre cherché est bleu clair". 
Contrairement à ce qu'on pourrait penser de façon intuitive, la réponse 
n'est pas $I(bleu) + I(clair) => 2 + 3 = 5 bits$ mais bien : 
$I(bleu) = \log{\frac{4000}{250}} = \log{16} = 4$.
La différence provient du fait que les informations sont dépendantes. 

Si l'on fait le même exercice mais avec I(rouge clair), cette fois-ci, 
on obtient $I(rouge clair) = I(rouge) + I(clair)$, et on peut en déduire 
que les informations sont indépendantes. En probabilités, on formule 
cela de cette façon : $P(rouge clair) = P(rouge)\times P(clair)$.


On peut regrouper des informations ensemble. Attention, car selon qu'elles 
sont dépendantes ou indépendantes, on obtient pas la même chose.

On peut dénombrer ainsi 3 cas de figure :
\begin{enumerate}
	\item L'information totale est inférieure à la somme de ses parties. 
	Se produit quand il y a dépendance, une information rend l'autre moins importante.
	\item L'information totale est égale à la somme de ses parties : 
	Les informations sont indépendantes
	\item L'information totale est supérieure à a somme des parties :
	Il y a dépendance, une information rend l'autre plus importante.
\end{enumerate}

Ceci permet d'en tirer une propriété multiplicative :
Prenons le cas d'un alphabet binaire, avec équiprobabilité de 0 et de 1. 
Chaque symbole est porteur de $\log{2}$ d'information, soit 1. 
Si le message est composé de n symboles, alors on obtient : 
$I = \log{2^{n}} = n \times \log{2} = n$ bits d'information.

En conclusion, on peut calculer l'information avec de longs messages 
aussi bien qu'avec des courts. 
Cette distance nous amène aussi à nous intéresser à un autre concept primordial : 
\textbf{l'entropie}.

\section{L'entropie revisitée}

L'entropie est un concept aussi fondamental pour la théorie de l'information 
que l'information. Du fait de son utilisation dans cette théorie, il se retrouve 
présent dans énormément d'autres domaines. Ainsi peut-on parler de 
l'entropie d'un style musical, d'une langue étrangère, etc.

L'information mesure plutôt la quantité "transmise", une production.
L'entropie, elle, se concentre plutôt sur le potentiel \textit{avant} 
la transmission du message, ce qui permet de comparer différents canaux, 
différentes sources, récepteurs, en fonction de leurs propriétés.

On note : 
$H = \sum_{i}p_{i}\times \log{\frac{1}{p_{i}}}$ où $p_{i}$ désigne la 
probabilité de l'événement i.

Elle peut sembler fort abstraite, mais appliquée à un exemple, elle prend 
tout son sens : 
Prenons le morse, avec $P(trait) = 0.75$ et $P(point) = 0.25$. 
La quantité d'information vaut $I(trait) = -\log{0.75} = 0.415$ bits, 
$I(point) = -\log{0.25} = 2$ bits. L'apparition d'un point pèse 
plus lourd que celle d'un trait. Par ailleurs, l'information 
d'un trait vaut moins qu'une unité, celle d'un point, deux.

Maintenant, si l'on prend un peu de hauteur, on imagine que cette expérience 
sera reproduite un nombre important de fois. Alors on obtient : 
$H = 0.75 \times 0.415 + 0.25 \times 2 = 0.811$ bits.
Intuitivement, cela représente la \textit{propension} d'un canal à 
émettre une certaine quantité d'information. C'est une information moyenne.

Dans le cas où il y a indépendance des probabilités, la somme se simplifie : 
$H = \log{\frac{N}{n}}$ bits. On peut en déduire que plus une distribution 
est équiprobable, plus l'entropie est forte.

\section{L'envers de l'information : la redondance}

La redondance diffère de la répétition, qui est un cas particulier de celle-ci.
Un terme pour la décrire serait plutôt "corrélation". En guise d'exemple, dans 
la langue française, un q est quasi systématiquement suivi d'un u, même si 
ces lettres ne sont pas les mêmes : ce n'est pas de la répétition, mais 
la lettre suivant q est moins difficile à deviner qu'après un a, ou un e. 
L'information du u après le q est donc peu informative.
La redondance évoque ce caractère peu informatif.\\

Afin d'optimiser l'information, on comprend que cette redondance doit être 
le plus faible possible. Cela peut être le cas dans un alphabet 
où chaque caractère est équiprobable. Dans ce cas, la redondance est nulle, 
et l'entropie atteint $\log(n)$ par symbole émis. 
La redondance d'une source est définie comme la différence entre 
l'entropie de la source et $\log(n)$.\\

Cependant, on ne cherche pas toujours à l'éviter. Ainsi, sur un canal bruyant, 
la redondance peut permettre de vérifier l'intégrité du message.

Les langues sont rarement les plus "efficaces" possibles. Ceci explique 
pourquoi chaque fois que la transmission d'un message est coûteuse, 
l'on cherche à mettre en place des systèmes plus courts, résumés, comme 
la sténographie, les abréviations, etc.\\

\subsection{Un nouveau sens pour les notions de bruit, d'équivoque et d'ambiguïté}

Le bruit est le mot qui définit la modification d'un message entre son 
émission et sa réception. Dans le cas d'un canal bruyant, on ne peut 
pas être sûr que le symbole reçu soit celui qui a été envoyé. On peut 
avoir une mesure du doute en probabilités. Dès lors, tout échange devient 
probabiliste. L'émetteur et le récepteur ne vont pas avoir accès aux 
mêmes informations. Ainsi, le récepteur ne peut calculer l'entropie 
qu'à partir des symboles reçus. C'est alors qu'interviennent les 
probabilités sous forme de "a sachant b", c'est à dire la probabilité que 
a ait été émis quand b a été reçu.\\

Le récepteur peut calculer l'entropie de cette façon : 
$H(a|b_{j})$. Si le canal n'est pas bruyant, alors l'entropie 
est nulle car b correspond toujours à a. Dès lors, on 
peut calculer $H(A|B)$, ce qui est appelé "l'\textit{équivoque} du canal".\\

La formule exacte de l'équivoque est : $H(A|B) = -\sum_{i,j}[p(a_{i}b_{j}).log(p(a_{i}|b_{j}))]$ 

(il y a une démonstration dans le livre, mais je me dispense d'en faire le résumé ici)\\

De façon symétrique, l'émetteur peut calculer l'incertitude que son message arrive 
à son destinataire. Cette mesure s'appelle l'ambiguïté, et sa formule est 

$H(B|A) = -\sum_{i,j}[p(a_{i},b_{j}).log(p(a_{i},b{j}))]$

et on peut montrer que $H(A|B) \leq H(A) + H(B)$ (il y a aussi une démonstration dans le 
livre dont je me dispense également)\\

Un cas particulier concerne le cas où ces deux quantités sont égales : 
$H(A|B) = H(A) + H(B)$.\\
Par conséquent, une autre quantité (la différence entre ces deux valeurs) a 
été définie : $T(A,B) = H(A) + H(B) - H(A,B)$
appelée \textit{transinformation}.\\

Cette valeur varie en fonction de la dépendance entre A et B. Si ces 
deux valeurs sont dépendantes, la transinformation augmente. \\

On peut en déduire : \\
$H(A,B) = H(A) + H(B|A)\\
H(A,B) = H(B) + H(A|B)\\
T(A,B) = H(A) - H(B|A)$\\
et les démonstrations sont dans l'ouvrage.\\

Dans le cas où la liaison entre A et B est parfaite, on a 
$T(A,B) = H(A) = H(B)$ car l'ambiguïté entre A et B est nulle.\\

Voici quelques illustrations afin de comprendre ce que signifient ces formules.
Mettons un émetteur A et un récepteur B. \\

Dans le premier cas, lorsque A essaie de transmettre un message 1, B reçoit 
avec une certaine probabilité ce même message. Idem pour un autre message. 
Dans ce cas, le bruit est nul.

Ce calcul se complique lorsque A essaie de transmettre un message, mais que 
B reçoit celui-ci, ou un autre. Dans ce cas, la transinformation diminue, 
et on dit que le canal est bruyant. Dans le pire des cas, on ne peut 
jamais être sûr du message reçu, la transinformation est de 0.\\

Un autre cas est celui où A essaie d'envoyer le msg 1, et B reçoit 
soit le msg 1 soit 2 avec une probabilité de 0.5. Si A envoie le msg 2, 
il n'est jamais reçu par B. 
Dans ce cas, on perçoit que quelle que soit l'idée exprimée par A, 
B l'interprète de la même façon. L'équivoque est donc très élevée, 
et l'ambiguïté peut prendre n'importe quelle valeur.\\

Dans le cas inverse, si A ne transmet que le msg 1 et que B reçoit 
soit le msg 1 soit 2, ces valeurs sont inversées : l'équivoque est quelconque, 
et l'ambiguïté est élevée. \\

Globalement, il y a équivoque lorsque le récepteur n'est pas aussi fin que l'émetteur, 
et il y a ambiguïté lorsque c'est l'émetteur qui n'est pas assez fin pour le récepteur.\\

\textit{Équivoque} : deux messages peuvent être compris de la même manière. \\
\textit{Ambiguïté} : un message peut être compris de plusieurs façons.\\

La capacité d'un canal est la transinformation maximale qu'on peut obtenir 
avec la loi de probabilité de la source la plus avantageuse possible. 
Des théorèmes issus de la théorie de l'information ont montré qu'une telle valeur limite 
est très utile, notamment car il existe toujours des codes permettant d'utiliser la 
totalité de la capacité d'un canal, même bruyant. Ces résultats ont été la source 
de nombreux développements de la théorie de l'information dans la communauté mathématique.\\

Malgré la force de ces valeurs, ce ne sont pas ces éléments qui ont rendu célèbre la 
théorie de l'information, mais des thèmes plus philosophiques comme la surprise, 
l'ordre, et la complexité.\\


\section{À la croisée de plusieurs concepts psychologiques et philosophiques essentiels}

\subsection{L'information comme réduction de l'incertitude}

Quoi que la théorie soit appelée "théorie de l'information", l'incertitude est 
un de ses thèmes centraux, si bien que certains ont même axé leur étude autour 
de celui-ci. Ils sont exactement opposés. En effet, 
le gain d'incertitude fait baisser l'information.
Il faut comprendre que la théorie de l'information de Shannon n'est valable 
qu'en considérant un ensemble fini et probabilisé. Dans le cas d'un ensemble 
infini, la théorie n'a aucun sens.\\

Un point soulevé concerne la mauvaise perception des probabilités par les humains. 
\textit{Ref citée : Harold W. Hake, The perception of frequency of occurrence 
and the development of "Expectancy" in humain experimental subjects"}, on y rappelle 
la très mauvaise intuition en la matière par les humains. Ceci permet d'en 
arriver à la notion psychologique de la "surprise". Celle-ci s'analyse en 
effet très bien à l'aide de la théorie de l'information.\\

\subsection{L'information comme résultat de la surprise}

Prenons le cas d'un professeur qui distribue un polycopié à une classe. 
À chaque distribution, les élèves répondent "merci". Cette intervention est attendue et 
très peu surprenante.

Mettons qu'à la fin du cours, un seul de ces élèves viennent dire "merci" au professeur. 
L'essence de cette intervention est bien plus lourde de sens ici, car elle n'est pas 
attendue, et le message est différent : dans le premier cas, on remercie par usage.\\

Dans le second, on exprime son remerciement pour la prestation.
Les chercheurs séparent deux notions pour exprimer la surprise : 
\textit{surprisal} est fonction de la quantité d'information telle que définie par la 
théorie de l'information. 
\textit{surprise} mesure quant à elle la composante psychologique de la surprise. \\

En guise d'exemple : si l'on lance un dé 5 fois, obtenir 5 faces a la 
même valeur de surprise que pile-face-pile-face, et pourtant, elle a une 
valeur de surprise. \\

Les événements surprenants sont plus riches d'information que les 
événements routiniers. Une faute d'orthographe pourra en effet soulever un 
questionnement : faute volontaire ? que signifie le message, etc. 
L'information est plus importante. Le fait de demander un message 
diminue également la surprise, ou l'information. \\

Bref, tout ceci se trouve à la limite de la théorie de l'information voire au delà. 
On voit qu'ici, les opérateurs mathématiques ne sont plus efficaces, et 
le vocabulaire de la théorie sert de réservoir de vocabulaire, mais 
ne sert pas vraiment d'outil. Cela n'a pas été souhaité par Shannon, 
et cela dépasse le cadre de la théorie.\\

Si l'on revient à la théorie en elle-même pour se recentrer sur son sens mathématique, 
on s'aperçoit que la théorie de l'information donne des outils pour décrire 
une notion primordiale dans les sciences dures et sociales : la complexité.


\subsection{L'information comme mesure de la complexité}

La théorie de la complexité est intuitive, elle est pourtant difficile 
à calculer. C'est Kolmogorov, et Gregory Chaitin qui sont considérés 
comme ses fondateurs (quoi que cette parenté soit discutée). \\

En guise d'exemple, prenons une chaîne composée de 0 et de 1, et qui alternent, ou 
une autre : \\
$01010101010101010101\\
01110010011011010010$\\
Si l'on souhaite prévoir une suite à ces chaînes, pour la première, on trouvera aisément, 
quant à la deuxième, on ne pourra rien prévoir. Mais sur le plan de l'information, 
elles sont deux chaînes de 20 caractères qui ont la même chance de tirer avec un tirage aléatoire 
de 0 et de 1. Il faut chercher plus loin avec le concept d’aléa pour voir ce qui les différencie. 

Ici intervient la notion de compressibilité : une suite qui peut se réduire à une suite plus 
courte sans perdre d'information n'est pas une suite aléatoire. La suite qui 
alterne les 0 et les 1 est compressible, la seconde ne l'est pas. Cela étant, le 
problème est moins simple qu'il n'y paraît. Mais dans tous les cas, il est possible 
de décrire la suite en un nombre minimal de bits. La \textit{complexité} est le 
nombre minimal de bits qui est nécessaire à la description de la suite, dans une machine de 
turing, par exemple. \\

En mathématique, globalement, ce qui est novateur est complexe, ce qui est répétitif est trivial.
Si l'on énonce deux exercices similaires, il faudra donner des détails pour le premier voire 
expliquer la procédure. Pour le deuxième, il suffit d'écrire "comme le premier". \\

\textbf{Evariste Galois} est un mathématicien célèbre du XX° siècle. qui a posé les bases 
de la théorie des groupes, on raconte qu'il a écrit sa théorie en une nuit avant sa mort. 
Cela est sans doute romancé, toujours est-il qu'il apparaît ici que des choses primordiales et 
importantes peuvent s'énoncer rapidement et sans redondance. Dans cet ouvrage, les démonstrations 
ne sont pas faites, et si le lecteur peut les faire lui-même, cela signifie qu'il aurait été
redondant de les écrire. Cela signifie que l'on se mette d'accord sur le contenu de l'alphabet de 
l'échange. Par exemple, pour transmettre une configuration de jeu d'échec, on peut tout décrire, 
ou décrire par une référence à une partie, et cela si l'on présuppose que le destinataire du 
message connaît cette référence. 

\subsection{L'information dans la problématique de l'ordre et du désordre}

Au début de cet ouvrage était mentionnée la proximité entre l'information et l'énergie.
Nous allons voir que ce lien est à manier avec précaution, mais qu'il s'est trouvé au 
centre de la théorie de l'intérêt lorsqu'une proposition de résolution 
du paradoxe de Maxwel a été proposé vers 1950.\\
Afin d'exposer ce paradoxe, il faut rappeler les bases de la thermodynamique.

\subsection{Second principe de la thermodynamique et démon de Maxwel}
Le second principe énonce qu'il y a une valeur (l'entropie) qui peut varier, 
mais toujours dans le même sens. Le second principe est directionnel, 
irréversible. Pour illustrer : prenons un réfrigérateur dont on ouvre la porte.
Les températures se mélangent naturellement, sans apport d'énergie. Pour retrouver la 
fraîcheur initiale, il faut un apport d'énergie.

\textbf{L'entropie} est la mesure physique du désordre. On postule que l'univers tend 
vers une entropie maximale, et qu'elle sera atteinte lorsqu'il y aura homogénéité.

Pour en revenir au démon de Maxwel, il s'agit d'une histoire où dans un grand récipient 
se trouvent deux compartiments. Le démon possède la possibilité d'ouvrir durant des temps brefs 
une porte entre les deux et de faire passer les molécules qu'il sélectionne, de sorte 
que le compartiment A va toujours se réchauffer, et le B refroidir. 
On voit que cette situation est un paradoxe si on accepte le fait qu'il ne coûte pas 
d'énergie d'ouvrir la petite porte au démon.
Les lois qui régissent la répartition des particules de gaz sont probabilistes, et 
on peut résumer l'entropie par la mesure des possibilités de mouvements 
des particules. \\

$S = k.\ln(W)$ est la formule physique de l'entropie. En fait, la formule 
est très proche de cette de l'entropie dans la théorie de l'information 
si bien que certains ont postulé que c'était l'information qui était nécessaire 
au démon pour savoir quand ouvrir la porte, idée postulée par Léon Brillouin. 
Mais les concetps restent différents, aussi a-t-il proposé d'adopter une autre 
terminologie : le terme \textbf{néguentropie} pour l'entropie qui concerne 
la théorie de l'information.

\subsection{Notion de néguentropie et paradoxe de l'information négative}

L'entropie a tendance à augmenter. Dans le meilleur des cas, elle stagne. 
La néguentropie, quant à elle ne fait que diminuer. 

Dans l'ouvrage, il y a des explications plus détaillées sur la différence entre les deux que je choisis de passer 
car les détails sont un peu ardus et ne me paraissent pas indispensables.
On y comprend que la théorie de l'information peut déboucher sur des choses bien 
plus vastes que la théorie de base, et que c'est sans doute cela qui lui a
permis son si grand succès.

\section{La théorie de l'information : Pour quoi faire ?}

\subsection{Une vocation d'origin toujours actuelle : La compression de données}




\end{document}
