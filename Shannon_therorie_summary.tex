%%%%%%%%%%%%%%%%%%%%%%%%%%%%%%%%%%%%%%%%%%%%%%%%%%%
%% LaTeX book template                           %%
%% Author:  Amber Jain (http://amberj.devio.us/) %%
%% License: ISC license                          %%
%%%%%%%%%%%%%%%%%%%%%%%%%%%%%%%%%%%%%%%%%%%%%%%%%%%

\documentclass[a4paper,11pt]{report}
\usepackage[T1]{fontenc}
\usepackage[utf8]{inputenc}
\usepackage{lmodern}
%%%%%%%%%%%%%%%%%%%%%%%%%%%%%%%%%%%%%%%%%%%%%%%%%%%%%%%%%
% Source: http://en.wikibooks.org/wiki/LaTeX/Hyperlinks %
%%%%%%%%%%%%%%%%%%%%%%%%%%%%%%%%%%%%%%%%%%%%%%%%%%%%%%%%%
\usepackage{hyperref}
\usepackage{graphicx}
\usepackage[english]{babel}
\linespread{1.6}

%%%%%%%%%%%%%%%%%%%%%%%%%%%%%%%%%%%%%%%%%%%%%%%%%%%%%%%%%%%%%%%%%%%%%%%%%%%%%%%%
% 'dedication' environment: To add a dedication paragraph at the start of book %
% Source: http://www.tug.org/pipermail/texhax/2010-June/015184.html            %
%%%%%%%%%%%%%%%%%%%%%%%%%%%%%%%%%%%%%%%%%%%%%%%%%%%%%%%%%%%%%%%%%%%%%%%%%%%%%%%%
\newenvironment{dedication}
{
   \cleardoublepage
   \thispagestyle{empty}
   \vspace*{\stretch{1}}
   \hfill\begin{minipage}[t]{0.66\textwidth}
   \raggedright
}
{
   \end{minipage}
   \vspace*{\stretch{3}}
   \clearpage
}

%%%%%%%%%%%%%%%%%%%%%%%%%%%%%%%%%%%%%%%%%%%%%%%%
% Chapter quote at the start of chapter        %
% Source: http://tex.stackexchange.com/a/53380 %
%%%%%%%%%%%%%%%%%%%%%%%%%%%%%%%%%%%%%%%%%%%%%%%%
\makeatletter
\renewcommand{\@chapapp}{}% Not necessary...
\newenvironment{chapquote}[2][2em]
  {\setlength{\@tempdima}{#1}%
   \def\chapquote@author{#2}%
   \parshape 1 \@tempdima \dimexpr\textwidth-2\@tempdima\relax%
   \itshape}
  {\par\normalfont\hfill--\ \chapquote@author\hspace*{\@tempdima}\par\bigskip}
\makeatother

%%%%%%%%%%%%%%%%%%%%%%%%%%%%%%%%%%%%%%%%%%%%%%%%%%%
% First page of book which contains 'stuff' like: %
%  - Book title, subtitle                         %
%  - Book author name                             %
%%%%%%%%%%%%%%%%%%%%%%%%%%%%%%%%%%%%%%%%%%%%%%%%%%%

% Book's title and subtitle
\title{\Huge \textbf{Invitation à la théorie de l'information, Emmanuel Dion}  
%\footnote{Notes de lecture} 
\\ 
\huge Notes de lecture
%\footnote{Notes : Arabella Brayer}
}
% Author
\author{\textsc{Arabella Brayer}
%\thanks{\url{www.example.com}}
}


\begin{document}

%\frontmatter
\maketitle

%%%%%%%%%%%%%%%%%%%%%%%%%%%%%%%%%%%%%%%%%%%%%%%%%%%%%%%%%%%%%%%
% Add a dedication paragraph to dedicate your book to someone %
%%%%%%%%%%%%%%%%%%%%%%%%%%%%%%%%%%%%%%%%%%%%%%%%%%%%%%%%%%%%%%%
%\begin{dedication}
%Dedicated to Calvin and Hobbes.
%\end{dedication}

%%%%%%%%%%%%%%%%%%%%%%%%%%%%%%%%%%%%%%%%%%%%%%%%%%%%%%%%%%%%%%%%%%%%%%%%
% Auto-generated table of contents, list of figures and list of tables %
%%%%%%%%%%%%%%%%%%%%%%%%%%%%%%%%%%%%%%%%%%%%%%%%%%%%%%%%%%%%%%%%%%%%%%%%
\tableofcontents
%\listoffigures
%\listoftables

%%%%%%%%%%%
% Preface %
%%%%%%%%%%%
\chapter{Introduction}

\section*{Le concept d'information}
Le terme "information" désigne une notion difficile à décrire de façon simple 
et sans emphase, 
ou sans user d'évidences qui n'apportent aucune information utile.
Pour ce faire, on peut s'inspirer de l'analogie entre 
l'information et l'énergie, notion aux multiples formes également.
D'autre part, remarquons que de tout temps, la plupart des inventions 
ont servi à maîtriser l'une ou l'autre : énergie, information.
Quelques exemples : la radio, le téléphone, l'informatique, etc.

\section*{Épistémologie}
Du point de vue de l'épistémologie, on peut également rapprocher 
l'information de l'énergie. On constatera alors que les deux ont été 
employées avant de savoir les définir de façon formelle. 
C'est avec la théorie de Shannon que l'information a acquis un sens précis, 
ainsi qu'une unité de mesure : le bit.
C'est la parution du livre de Shannon en 1948 qui marque ce tournant, 
et qui restera dans l'histoire des sciences du XX$^{o}$. Dès ce moment, 
un nombre important de publications sortent à ce sujet, 
et la recherche clarifie son discours. 

Actuellement, la densité de travaux s'est certes un peu tarie, 
néanmoins l'ensemble de ces travaux sont rassemblés derrière l'expression 
"théorie de l'information" (ainsi que "théorie de la communication"
\footnote{"Théorie de la communication" est une expression qui désigne 
la même chose strictement, contrairement à ce que laisse entendre son 
nom. Shannon lui-même aurait préféré l'usage de l'expression 
"théorie de l'information".}
) et est largement reconnue.

Parmi les théories existantes en sciences, on pourrait trouver des éléments 
similaires entre la théorie de l'information et la théorie des jeux : 
double composante mathématique et conceptuelle, ainsi qu'une large 
diffusion. D'ailleurs, même si le lien entre ces deux théories ne 
saute pas à la conscience, elles entretiennent des relations, 
qui seront détaillées plus tard.

\section*{Utilisations de la théorie}
% You might want to add short description about each chapter in this book.
La théorie de l'information a été vue de façon différente dans la science : 
ainsi a-t-elle apporté à plusieurs domaines, tels que la biologie, la psychologie, etc. 
Mais son caractère "généraliste" lui a "permis" d'être largement citée en philosophie. 
Il s'agirait plutôt d'un emploi abusif. On pourrait tenter de réduire 
la théorie de l'information à quelques opérateurs mathématiques, 
déjà connus, mais réunis dans cette théorie. On peut également la voir 
comme une théorie primordiale pour le XX$^o$ siècle.

\textbf{Problématique} :
Ce débat a-t-il lieu d'être ou pourrait-on imaginer que ces 
deux propositions ne se rassemblent ? 

%%%%%%%%%%%%%%%%
% NEW CHAPTER! %
%%%%%%%%%%%%%%%%
\chapter{La théorie de l'information : une théorie transversale au cœur de la science moderne}

%\begin{chapquote}{Author's name, \textit{Source of this quote}}
%``This is a quote and I don't know who said this.''
%\end{chapquote}

\section{Section heading}

La théorie de l'information ne s'intéresse absolument pas à la 
signification, au sens, contrairement aux autres théories en 
communication, focalisées sur cet aspect.

\subsection{Lorem ipsum dolor sit amet, consectetur adipiscing elit.}
Lorem ipsum dolor sit amet, consectetur adipiscing elit. 

\subsection{Lorem ipsum dolor sit amet, consectetur adipiscing.}
Lorem ipsum dolor sit amet, consectetur adipiscing elit. 

\subsection{Lorem ipsum dolor sit amet}
Lorem ipsum dolor sit amet, consectetur adipiscing elit.

In hac habitasse platea dictumst. 

\subsection{Lorem ipsum dolor sit amet, auctor et pulvinar non}
Lorem ipsum dolor sit amet, consectetur adipiscing elit. Duis risus ante, auctor et

In hac habitasse platea dictumst. Nullam turpis erat, porttitor ut pretium ac, 

\section{Another section heading}
Lorem ipsum dolor sit amet, consectetur adipisicing elit, sed do eiusmod tempor

%%%%%%%%%%%%%%%%%%%%%%%%%%%%%%%%%%%%%%%%%%%%%%%%%%%%%%%
% Sample table                                        %
% Source: www1.maths.leeds.ac.uk/latex/TableHelp1.pdf %
%%%%%%%%%%%%%%%%%%%%%%%%%%%%%%%%%%%%%%%%%%%%%%%%%%%%%%%
\begin{table}[ht]
\caption{Sample table} % title of Table
\centering % used for centering table
\begin{tabular}{c c c c}
% centered columns (4 columns)
\hline\hline %inserts double horizontal lines
S. No. & Column\#1 & Column\#2 & Column\#3 \\ [0.5ex]
% inserts table
%heading
\hline % inserts single horizontal line
1 & 50 & 837 & 970 \\
2 & 47 & 877 & 230 \\
3 & 31 & 25 & 415 \\
4 & 35 & 144 & 2356 \\
5 & 45 & 300 & 556 \\ [1ex] % [1ex] adds vertical space
\hline %inserts single line
\end{tabular}
\label{table:nonlin} % is used to refer this table in the text
\end{table}

Duis aute irure dolor in reprehenderit in voluptate velit esse cillum dolore eu fugiat nulla pariatur. Excepteur sint occaecat cupidatat non proident, sunt in culpa qui officia deserunt mollit anim id est laborum. \\ Lorem ipsum list:
\begin{itemize}
\item Mauris sit amet nulla mi, vitae rutrum ante.
\item Maecenas quis nulla risus, vel tincidunt ligula.
\item Nullam ac enim neque, non \emph{dapibus} mauris.
\end{itemize}

\noindent Lorem ipsum dolor sit amet, consectetur adipiscing elit. Duis risus ante, auctor et pulvinar non, posuere ac lacus. Praesent egestas nisi id metus rhoncus ac lobortis sem hendrerit. Etiam et sapien eget lectus interdum posuere sit amet ac urna\footnote{Lorem ipsum dolor sit amet, consectetur adipiscing elit. Duis risus ante, auctor et pulvinar non, posuere ac lacus.}:

\subsection{Lorem ipsum dolor sit amet, consectetur adipiscing elit.}
Lorem ipsum dolor sit amet, consectetur adipiscing elit. Duis risus ante, auctor et pulvinar non, posuere ac lacus. Praesent egestas nisi id metus rhoncus ac lobortis sem hendrerit. Etiam et sapien eget lectus interdum posuere sit amet ac urna. Aliquam pellentesque imperdiet erat, eget consectetur felis malesuada quis. Pellentesque sollicitudin, odio sed dapibus eleifend, magna sem luctus turpis, id aliquam felis dolor eu diam. Etiam ullamcorper, nunc a accumsan adipiscing, turpis odio bibendum erat, id convallis magna eros nec metus. Sed vel ligula justo, sit amet vestibulum dolor. Sed vitae augue sit amet magna ullamcorper suscipit. Quisque dictum ipsum a sapien egestas facilisis. 

\subsection{Lorem ipsum dolor sit amet, consectetur adipiscing}
Lorem ipsum dolor sit amet, consectetur adipiscing elit. Duis risus ante, auctor et pulvinar non, posuere ac lacus. Praesent egestas nisi id metus rhoncus ac lobortis sem hendrerit. Etiam et sapien eget lectus interdum posuere sit amet ac urna. Aliquam pellentesque imperdiet erat, eget consectetur felis malesuada quis. Pellentesque sollicitudin, odio sed dapibus eleifend, magna sem luctus turpis, id aliquam felis dolor eu diam.

\end{document}
